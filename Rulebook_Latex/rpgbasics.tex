\section{Basics of Playing}

\par
\getTitle\, is best for 3-6 players, plus a \gm . Each game session will probably take from two to four hours. Games can take as many sessions as your groups wants to play.

\par
If you already know how to play tabletop RPGs, you can probably skip this section. If you don't, welcome! We hope this will be far from the last one you'll play.

\par
The game is a collaborative story told by the players with the help of the \gmLong\, (\gm). The players navigate characters through the universe that the \gm\, creates. The players don't play against each other, and definitely don't play against the \gm , nor does the \gm\, try to act as an enemy of the players. They all negotiate the story together.

\par
Before the game starts, players \hyperlink{Character Creation}{create a character} using a character creation sheet.

\par
The \gmLong (\gm , for short) creates a universe of planets, ships, and locations, and people. They sit at the head of the table, and failitate the players and their \pc s interacting with this universe. The \gm\, acts as every person and environment that the \pc s come across.

\par
The \gm\, will have to plan out the planets, cities, ships, people, and aliens encountered by the \pc s. They should arrive at each game session prepared for the next places the players may go. Ability to plan, but be flexible with those plans, is important. Having a stockpile of extra places and people ready will help tremendously when players go someplace unexpected. More on running a game is outlined in the \getTitle\,\gm 's Book.

\subsection{The Universe}

\par
The universe of \getTitle\, is a vast one.

\subsection{The People}

\par
Write about people and types of species.

\subsection{The Technology}

Write about realistic-ish technology.

\subsection{The Aliens}

Write about very unusual aliens.