%Explaination on how skills and abilities work.

\section{\skillC s}

\par
\skillC s are specific areas of expertise that help you complete actions by giving a helpful bias to certain rolls. When you roll for an action, you can apply \textbf{one} of your \skill s to that roll, as long as the \gm\, agrees that your \skill\, is relevant to the action you're trying to do.

\par
Whenever your roll, you get one \xp . You can acquire a \skill\, by spending the \xp\, amount required to get it. For each skill, it costs 1 \xp\, for level 1, an additional 2 \xp s for level 2, and finally 3 more \xp s to finish it off with the third level. You can do this at any time, during the game, before or after the session, or sitting at home in between sessions. The types of situations a \skill\, can be used are written by the \skill 's name. Also, the amount of the \modifier\, is listed.

\par
You can also 'level up' \skill s. The cost of additional levels is listed in that \skill 's description. Leveling up \skill s means it biases your roll more strongly. A \skill\, can be upgraded twice, meaning that it can be as high as level 3. Check the \skill 's description, though, since level 3 doesn't necessarily mean that it gives a +3 or -3 bias.

\par
Remember, the bias given by the \skill 's \modifier\, does not change based on your \both\, number. If you are heavily towards a \feelings\, direction, a \lasers\, skill \modifier\, is just as strong. Your overall odds of success on a \lasers\, roll is still low, but for certain \skill s it may still be higher than it would be without the \skill .

\subsection{\abilityPC}
\abilityPC\, are like skills, in that they cost \xp\, but instead of aiding a roll, they allow you to do something you could not before. \abilityPC\, cost 5 \xp\, per level, across the board. \skillC s are about increasing the chances of success for things you already know how to do, but \abilityP\, are about doing new types of things.