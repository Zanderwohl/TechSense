\section{Session Zero}
\par
So you're ready to play. Your characters need to get to know each other (and maybe your players, too). The \gm\, should block out an hour or so for everyone to get on the same page. Start by introducing the world, the ship, and what kinds of missions or goals your organization has. What are the "given circumstances"? What is the recent history of this this world? What's the organization you're part of like? Try to give a sense of the vastness of this universe, and the strangeness of its inhabitants.

\par
Next, the characters should introduce themselves to each other. After giving a real-life name, switch into the character, talking as if you \textit{are} that charcter. While not essential, a character voice helps differentiate your character's opinion and questions from your own. Before talking about what your character does, talk about who they are. Where are they from? What do they look like? What's their cultural background? What skills do they bring to this starship? Why did they want to become an astronaut? Highlight things that are important to the character.

\paragraph{An Example Introduction}
I'm Jane. My character's name is Natalie Cook. [Here, you can switch into your character.] \textit{I'm a nuclear engineer from the South Pacific League. My parents worked on Earth's space elevator. I'm a woman who looks South Pacific and have short black hair that I keep in a bun. I usually wear the standard jumpsuits, even on shore leave. Since I grew up around the kids of oher engineers, I know a English-Polynesian-German creole unique to the area around the project. I work on the ship's propulsion systems, and became an engineer specifically to join a starship crew. I became an astronaut because I want to discover how other planets and cultures solve engineering problems. I play board games in my free time and brought chess in my personal items. Hit me up for a match!}

\paragraph{}
The astronauts (the whole crew of the ship) would have gone through several months of training with each other. They know each other well, trust each other, and are open with feelings and concerns. Maybe they don't know each other's personal lives in detail, but there should be no loners. After all, a starship is only so big. A broody or antisocial person would have been weeded out pretty early on. Have everyone discuss what friendships or friendly rivalries were developed during training and mission simulations. Then, have characters introduce and describe these relationships with each other.

\paragraph{An Example Relationship}
\textit{Debola and I are the two main pilots on this ship. During training, we spent a lot of time competing for high evaluations on simulations. Sometimes we would play the same senarios dozens of times to shave the smallest amounts of fuel off of docking or landing costs. We relentlessy mock each other's skills, but go out for a drink together once we're done for the day. Debola's probably my closest friend on the ship.}

\paragraph{}
Finally, the \gm\, should introduce the players to the first mission. They'll be able to choose which missions to take on later, but a short, simple one to begin will help give a sense of how the game goes. Play through this in about an hour, ending on a cliff-hanger, like a distress call, a strange ship exiting lightspeed nearby, or a failure being discovered on the ship. A sense should be given that something is about to happen at the beginning of the next session.