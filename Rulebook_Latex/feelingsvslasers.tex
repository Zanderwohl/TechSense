\section{\feelingsvslasers}
\par
There is only one stat in \getTitleShort, and it's not \textit{terribly} important. It's the \feelingsvslasers\, axis. It's pretty simple: the closer you are to \feelings, the better you are at interpersonal relationships, diplomacy, talking, etc., and the closer you are to \lasers, the better you are at using technology, navigating, piloting, and so on. But both are on the same axis, and your character keeps track of just \textit{one} number. We'll call this your \textit{\both\, number.}

\par
Actions work like this: roll a twelve-sided die (a.k.a a \textit{d12}). If you're rolling for a \lasers\, action, then you succeed if you roll your \both\, number or higher. But if you're rolling for a \feelings\, action, you succeed if you roll your \both\, number or \textit{lower}. So the better you can roll on \lasers, the worse your ability to succeed on \feelings\, rolls, and vice-versa.

\par
You're not locked in to a single \both\, number, though. Every time you take an action that requires a \lasers\, roll, your \both\, number goes down by one (down to a minimum of \lowerlimit), making a future \lasers\, success more likely, and you gain one \lasersxp. Every time you take an action that requires a \feelings\, roll, your \both\, number goes \textit{up} by one (up to maximum of \upperlimit), making a \feelings\, success more likely, and you gain one \feelingsxp. So you may bounce back and forth along the \feelingsvslasers\, scale, depending on what actions you take. Up to \upperlimit\, then down to \lowerlimit\, again. You'll get better and worse and different rolls over time.  If you want to focus on one, you'll have to take both types of actions equally to keep your advantage at one end.

\par
But here's what's important: you don't lose the \modifier s that you acquire from spending your \lasersxp s and \feelingsxp s whether that be +1 or +2 or -1 (remember, this isn't a bad thing for \lasers\, rolls) you might get on rolls while using \textit{that} \skill . So even though you, normally playing with \lasers , may move toward the \feelings\, side over time, making rolls harder, your specialized \lasers\, \skill s will still push your \lasers\, rolls toward success.

\par\textit{Note: Abilities are not affected by the \pc `s current \both\, number at all. You don't roll on abilities.}

\par
The other effect of this system is that it forces you to round out your abilities a little. Starting at an extreme of 2, you can only acquire nine \lasersxp s before you have to get at least one \feelingsxp, to acquire another \lasersxp .

\subsection{Failure}
If you roll, and fail, then something goes wrong. It's up to the \gm\, to determine what goes wrong. The \gm\, should use their best judgement to determine the severity of what happens. If, for example, a shuttlecraft's instruments are out, a player may try to land by the seat of their pants, and choose to roll with a pilot \skill\, aiding that roll. If their \both\, number is 4, and their pilot \skill\, is 2, but they roll a 1, they get a total of 3 and they fail. However, it might be cruel to TPK (\textit{total-party kill; killing every player's character}) and end the game right then.

\par
Instead, the \gm\, may choose some consequences. Perhaps the shuttle lands far off-course, and has its landing gear damanged on touchdown, preventing the players from taking off until they repair it. Perhaps they crash-land in someone's barn, making the locals upset. Depending on the severity of the failure, and how much pity the \gm\, wants to take on their hapless players, the consequences could be lesser or greater.

\par
Be sure not to let a failure halt all action; that kills the pacing of the story and morale of your players. Players \textit{do} enjoy a challenge when they're forced to fight harder. Making success impossible will force them to give up and disengage with the game.