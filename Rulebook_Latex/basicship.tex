\newcommand{\shipname}{Stone's Throw}
%similar to the Antares Dawn battlecruiser as shown on project rho - realistic designs page 1

\section{The \textit{\shipname}}
\shipname\, is the basic ship for \getTitleShort . Other ships are outlined in the \shipBookTitle , but this is the basic player's ship.

\subsection{Description}
\par
\textit{Stone's Throw} is a Torch Ship. A ''Torch'' ship is any ship with an engine that is unreasonably powerful - that is, it can get around the solar system in a matter of weeks, at unreasonably low fuel cost.

%with a radius of 50 meters, and a height of 40 meters, the radius of the bottom is 20 meters, and the volume is 58,643 cubic meters, and the surface area is 6,283 square meters.
\par
The \textit{Stone's Throw} is a truncated sphere, with its engines and landing gear on the flat underside. The sphere is 50 meters in diameter, and there's about 58,500 cubic meters of space encompassed by the sphere. About 12,000 cubic meters are taken up by the reactor and engine assembly, plus an additional 5,000 cubic meters for other structural elements, leaving about 41,000 cubic meters of space for everything else. This feels cramped for the 300-person crew compliment. The maximum capacity of the ship is 400 people.

\subsection{Daily Life}
\par
Each person gets about 8 cubic meters of space to call their own. This seems uncomfortably tight for a bed and personal cabinets, but in null g every surface becomes a wall, a floor, \textit{and} a ceiling.

\subsection{Energy Source}

\subsection{Engines}

\subsection{FTL Engine}

\subsection{Structure and Hull}

\subsection{Armaments}

\subsection{Sensors}

\subsection{Ansible}

\subsection{Shuttles}

\subsubsection{Oxygen}
Oxygen is generated via a hydroponic algae that feeds off the heat of the Torch reactor.

\subsection{Other Ships}
Here is a table of other ships based off the \shipname . Fill in details, but here are their basic stats:

\begin{tabular}{|l | c | r | r | r | r | r |}
\hline  Name & Diameter & Dry Mass & Wet Mass & Max Mass & Power & Accel.\\\hline
        The Hammer & 70 m &  120 kT &  160 kT &  185 kT &  idk &  3g\\
        The Spritely & 30 m &  40 kT &  55 kT &  62 kT &  idk &  6g\\
        The Cheat & 30 m &  40 kT &  60 kT &  120 kT &  idk &  6g\\
\hline\end{tabular}