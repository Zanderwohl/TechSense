\section{Other Information}

\subsection{The Game}

\par
The game, plus all its rulebooks and source files, can be downloaded gratis at <\website> .

\subsection{Inspiration and Further Reading / Playing}
\par
This game was inspired by several other games, as well as books.

\paragraph{Lasers and Feelings}
by John Harper (Twitter: \href{https://twitter.com/john_harper/}{@john\_ harper}). This is a very lightweight RPG system that fits on a single page; its lasers and feelings axis inspired the creation of this game, after we played nearly an entire Apocalypse World campaign without rolling on about half our stats more than a couple times. When first under development, this game's name was even "Lasers and Feelings with Advancement".

\paragraph{The Left Hand of Darkness}
by Ursula K. Le Guin. An envoy for humanity is sent to a planet known as Winter to establish first contact. He finds a strange species of human that is biologically androgynous, and free of gender, causing him to re-examine assumptions he has made about his culture and himself. This story inspired the strange types of humanoid aliens found in this game.

\paragraph{The Rolling Stones}
by Robert Heinlein. A set of genius twins decide to use the money they made from an invention to buy a ship and wander the solar system. Unfortunately for them, their family wants to tag along. Likely influenced the Star Trek episode \textit{The Trouble with Tribbles}.

\paragraph{Space Cadet}
by Robert Heinlein. A boarding-school story set in space. A handful of space cadets find themselves on their own when asked to investigate some strange occurances on Venus. Features a flight computer controlled by gears, where a cam must be cut for each planet according to its gravity and atmosphere, to correctly control automatic landing.

\paragraph{Citizen of the Galaxy}
by Robert Heinlein. A young slave boy is taken in by space traders after his adoptive father is executed for being a spy. He travels between many strange and incomprehensible cultures before arriving on the one which to him is strangest of all - Earth.

\paragraph{The Moon is a Harsh Mistress}
by Robert Heinlein. The moon, a former prison colony for Earth, grows restless at the poor treatment of 'loonies' by Earth. They start a rebellion, opting to lob rocks at Earth via an enormous railgun, which impact the surface like atom bombs thanks to the gravity well. Shows an enormous complex of computers becoming sentient emergently, who is probably the most interesting character in the book. Also features the sex-fantasties-disguised-as-culture endemic to Heinlein's adult novels, and even tries to defend them, though not graphically. {\tiny At least it's not straight masturbatory material like \textit{Stranger in a Strange Land}.}

\paragraph{Star Trek: The Original Series}
One of the early science fiction television serials, Star Trek has both fun concepts and a healthy helping of ham. About half the episodes are watchable to a modern viewer without a literary background. The episodic format of unusual creature encounters, as well as the existence of space colonies from before the invention of lightspeed that were forgotten may have come to \getTitle\, from this series.

\paragraph{The Forbidden Planet}
An early sci-fi movie which Gene Roddenberry credited for inspiring Star Trek. Features a totally invisible monster whose motivations are a classic sci-fi twist. Also has a classic "father and daughter alone on a desert planet with secrets to hide and maybe they'll murder people to keep it that way" which is another fun one.

\subsection{The Book}
\par
\textcopyright 2019 \zandy . Rights and Licensing explained in the \hyperlink{Licensing}{Licensing} section. 

\par
This book was written by \zandy\, and edited by \sam . It is typesetted in \LaTeX .

\subsection{Modifying the Book}

\par
Because \getTitle\, is licensed under the \hyperlink{Licensing}{GNU General Public License}, you are free to modify, redistribute, and even sell (as silly as that would be) \getTitle , with or without modifications, without asking for permission from the authors or paying any kind of licensing fee.

\par
The authors ask that you would attribute credit to them, just as they have attributed credit to the works and systems that inspired the creation of this game.

\par
The source \LaTeX\, files can be found at 
\par
<\website >
\par
and downloaded gratis, just like the rest of the game.

\subsection{Contributing}

\par
Because this is a Free project, contributions are welcome.

\par
If you'd like to contribute, you can write new sections, campaigns, or additions, and submit them via a pull request on the \getTitle\, \hyperlink{https://github.com/Zanderwohl/TechSense}{github repository}. For modifications, please modify the \LaTeX\, files. For new books, submit the book as a new directory with the book, containing the .pdf and a subdirectory labelled "src" containing the \LaTeX\, files.

\par
New campaign and reference books, and reference charts would be very welcome.

\subsection{Licensing}
\par
\hypertarget{Licensing}{This game and book, as well as other associated books bundled with it are part of \getTitle .}

\par
\getTitle\, is free: you can redistribute it and/or modify it under the terms of the GNU General Public License as published by the Free Software Foundation, either version 3 of the License, or (at your option) any later version.

\par
\getTitle\, is distributed in the hope that it will be useful, but WITHOUT ANY WARRANTY; without even the implied warranty of MERCHANTABILITY or FITNESS FOR A PARTICULAR PURPOSE. See the GNU General Public License for more details.

You should have received a copy of the GNU General Public License along with \getTitle . If not, see <\href{https://gnu.org/licenses/}{https://gnu.org/licenses/}> or write to the Free Software Foundation Inc., 51 Franklin Street, Fifth Floor, Boston, MA 02110-1301, USA.