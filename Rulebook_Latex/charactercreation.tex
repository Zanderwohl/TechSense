\hypertarget{Character Creation}{}
\section{Character Creation}

\subsection{Fiction First, then Mechanics}

\par
It can be tempting to create a spreadsheet of \skill s and slap on a face later. This can help with the numbers of your character, but you will have a hard time interacting with other players or \npc s if you don't know \textit{who} your character is.

\par
It's better to come up with an interesting character, full of story and life, then choose the \skill s and \abilityP that fit that character. Since you will be changing your \skill s and \abilityP frequently, you shouldn't define your character by those things alone.

\par
Here's an example backstory:

\paragraph{An Example Backstory}
\textit{Maverick is a human from Terra. He grew up in the state of Iowa in the American Union. He is on the ship to be an envoy, with a specialization in cultural traditions. He hopes to find his uncle who went off with an interstellar mining guild fifteen years ago.}

\par
This example includes a number of good elements. It includes a species and heritage, providing a cultural background. It also has a job you can build skills around. Finally, it has some abstract personal goal unrelated to the character's job. Try to keep it short, but rich. Specific details are good, but don't bog down your character with too much. Create only what you need to play out the story, and don't be afraid to discover more as you go.\footnote{If you have any formal acting training, what we're doing here is exploring the \textit{given circumstances}.}

\par
\subsection{Superobjective}
It can also be useful to have an overarching goal or development that your character may have during the story arc. This major goal is called a \textit{superobjective}\footnote{Again, a term taken from the theatre.}. This superobjective may be something like \textit{reinvigorate my love and wonder for the universe} or \textit{find my long-lost brother} or even \textit{exact revenge on the asteriod-mining mogul who cost my mother her life}. These goals should carry an emotinal component, because they are \textit{important} to your character. Don't force your character to go through these changes or complete their superobjective, but take opportunities to explore them. Your character doesn't even have to be aware they have this need or want, but should have a tendency toward completing it. If you make your \gmLong\, aware of your character's superobjective, they may integrate more opportunities or encounters into the story to help your character along.

\par
Another note: A lot of stories are about change and irony. Maybe the completion of your superobjective doesn't satisfy them like they thought it would. Maybe they change their standards and learn to be happy with never knowing some truth, or never getting what they wanted. Maybe they do get it, but it costs them \textit{everything} else important to them like other friends, honor, or their own identity.\footnote{Also known as the principle of \textit{Yes, But...}}

\par
To take our \textit{exacting revenge on the asteroid-mining mogul} objective from earlier, and play it out, perhaps your charcter realizes they cannot bring themselves to commit murder, or refuse to save him when he's in danger, or perhaps your character \textit{does} kill him, but ends up leading a violent gang and falling through the morals-event-horizon, never to be the same.

\subsection{Switching Characters}

\par
You may also switch characters frequently, so you don't have to be locked in to one forever. At the end of a session, you can switch to any character on your crew, or who is a (semi-)permanent resident of your ship. Be sure to work with other players so that your new character doesn't fulfill a need that another character already fills.

\par
Before you come to the next session to dive into this new character, make it your own. You may know a few things about this character, or even quite a bit, depending on how much you interacted with them when they were an \npcShort . Try to keep them true to what you've learned so far, but also add backstory, beliefs, experiences, history, and motivation to them, so that you can play them fully. The \gm\, may also choose to give you certain notes about this person to help you.

\par
\textit{Note to \gm : don't make a crew member \npcShort\, a critical piece to a plot, just in case a player switches to that \npcShort . If you must have a traitor or mole, make them either a temporary traveller joining the ship for a short time, or some other member of the faction that this ship is part of, to be transported or taken care of for a short time. This way, the critical \npcShort\, can play their role, but not get in the way of players.}

\subsection{The Crunch}

\par
Once you have a character in mind, it's time to get into the numbers, baby!
\begin{enumerate}
	\item Choose a \both\, number. \\
		\textit{The higher the number, the more likely a \feelings\, roll will suceed, and the lower the number, the more likely a \lasers\, roll will suceed.}
	\item Choose a \skill\, to max out.\\
		\textit{Presumably, your character is an expert at something! That's why they were chosen to be on this ship. Pick any skill, and give yourself the maximum level.}
	\item Choose two more \skill s at level one. \\
		\textit{You have other skills and training. Choose two things to have a basic proficiency at.}
\end{enumerate}

\paragraph{Our Example from Earlier}
\textit{Let's continue creating Maverick. Maverick now needs a \both\, number, \skill s, and \abilityP . Since he's an envoy, he should probably start with a high \both\, number, so he can roll better on \feelings\, rolls. Let's go with 9. We should also mostly give him \feelings -based \skill s. Let's give him \hyperlink{SkillEnvoy}{Envoy} as his maxxed-out skill. Let's also give him \hyperlink{SkillAnthropologist}{Anthropologist} at level 1, since that fits his mission well. It might be wise to also give him a \lasers\, \skill , too. Let's pick \hyperlink{SkillUserInterfacer}{User Interfacer} (his second level-1) because it helps round him out, but also fits with his character as someone who figures out strange and alien civilizations.}

\par
As you take actions, remember to gain one \xp\, every time you roll. At the beginning and end of sessions, you can spend your \xp\, on gaining \skill s and \abilityP . How to aquire new \skill s and \abilityP\, is outlined in \hyperlink{skills}{the section on \skill s and \abilityP} .

\par
That's pretty much it for basic character creation. Add more backstory to taste.