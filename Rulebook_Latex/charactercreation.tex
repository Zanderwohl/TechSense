\section{Character Creation}

\subsection{Fiction First, then Mechanics}

\par
It can be tempting to create a spreadsheet of \skill s and slap on a face later. This can help with the numbers of your character, but you will have a hard time interacting with other players or \npc s if you don't know \textit{who} your character is.

\par
It's better to come up with an interesting character, full of story and life, then choose the \skill s and \abilityP that fit that character. Since you will be changing your \skill s and \abilityP frequently, you shouldn't define your character by those things alone.

\par
uhh....

\paragraph{An Example Backstory}
\textit{Maverick is a human from Terra. He grew up in the state of Iowa in the American Union. He is on the ship to be an envoy, with a specialization in cultural traditions. He hopes to find his uncle who went off with an interstellar mining guild fifteen years ago.}

\par
This example includes a number of good elements. It includes a species and heritage, providing a cultural background. It also has a job you can build skills around. Finally, it has some abstract personal goal unrelated to the character's job.

\subsection{Switching Characters}

\par
You may also switch characters frequently, so you don't have to be locked in to one forever. At the end of a session, you can switch to any character on your crew, or who is a (semi-)permanent resident of your ship. Be sure to work with other players so that your new character doesn't fulfill a need that another character already fills.

\par
Before you come to the next session to dive into this new character, make it your own. You may know a few things about this character, or even quite a bit, depending on how much you interacted with them when they were an \npcShort. Try to keep them true to what you've learned so far, but also add backstory, beliefs, experiences, history, and motivation to them, so that you can play them fully. The \gm\, may also choose to give you certain notes about this person to help you.

\par
\textit{Note to \gm : don't make a crew member \npcShort\, a critical piece to a plot, just in case a player switches to that \npcShort . If you must have a traitor or mole, make them either a temporary traveller joining the ship for a short time, or some other member of the faction that this ship is part of, to be transported or taken care of for a short time. This way, the critical \npcShort\, can play their role, but not get in the way of players.}

\subsection{The Crunch}

\par
Once you have a character in mind, it's time to get into the numbers, babey!
\begin{enumerate}
	\item Choose a \both\, number. \\
		\textit{The higher the number, the more likely a \feelings\, roll will suceed, and the lower the number, the more likely a \lasers\, roll will suceed.}
	\item Choose a \skill\, to max out.\\
		\textit{Presumably, your character is an expert at something! That's why they were chosen to be on this ship. Pick any skill, and give yourself the maximum level.}
	\item Choose two more \skill s at level one. \\
		\textit{You have other skills and training. Choose two things to have a basic proficiency at.}
\end{enumerate}

\paragraph{Our Example from Earlier}
\textit{Let's continue creating Maverick. Maverick now needs a \both\, number, \skill s, and \abilityP . Since he's an envoy, he should probably start with a high \both\, number, so he can roll better on \feelings\, rolls. Let's go with 9. We should also mostly give him \feelings -based \skill s. Let's give him \hyperlink{SkillEnvoy}{Envoy} as his maxxed-out skill. Let's also give him \hyperlink{SkillAnthropologist}{Anthropologist} at level 1, since that fits his mission well. It might be wise to also give him a \lasers\, \skill , too. Let's pick \hyperlink{SkillUserInterfacer}{User Interfacer} (his second level-1) because it helps round him out, but also fits with his character as someone who figures out strange and alien civilizations.}

\par
That's pretty much it for basic character creation. Add more backstory to taste.