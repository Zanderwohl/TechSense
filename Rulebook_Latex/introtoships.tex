



\section{Ships}

\par
If you want to travel the stars, it's likely you're going to want a ship to get around in. Most ships in \getTitle\, are \textit{torch ships}. A torch ship is a ship that, within the universe, still obeys normal physics like orbital mechanics (that is, doesn't fly around like an airplane), but its engines are absurdly powerful.

\par
Despite many attempts to be as realistic as reasonable, faster-than-light travel (\textit{FTL}) is also possible in \getTitle . We made this concession only to allow for wider exploration. Near-light travel would allow travel to other solar systems within a lifetime, but only to those within the ship. To the outside universe, decades would go by. So we have faster-than-light drives.

\par
Ships have many differences between them, so here is an overview of what ships can have.

\subsection{Maneuvering}

\subsubsection{Burns}
\par
A \textit{burn} is any time an engine turns on, and causes the velocity of the ship to change. All maneuvers are made through burns. Burns cost fuel, of course.

\subsubsection{Ballistics}
\par
Ballistics are traditional rocket maneuvering using optimized burns. These burns can be quite complex\footnote{Not rocket science complex, but \textit{almost}.} thanks to the unintuitive nature of orbits. You can't just point the direction you want and \textit{go}. You've got to calculate how your velocity will change over time as planets and moons pull on your ship.

\par
Fortunately, we've included several charts for simple calculations, and program tools to do some more complex stuff... to a point. If you already know about orbital mechanics and just \textit{love} calculating plane changes and phase matching then go ahead.

\par
If you don't love all that stuff just use the charts or do torch burns for everything. What's a torch burn? Glad you asked...

\subsubsection{Torch Burns}
\par
You know what we just said about not being able to point where you want and just \textit{going?} Actually, you can, it's just absurdly expensive. But as we mentioned, our ships can be absurdly powerful. A Torch burn is also called a \textit{burn-flip-burn}. You point to your target, burn toward it until you reach the halfway point, flip 180\textdegree , then burn until you come to a stop.

\par
The key is that since there's no brakes or friction in space, you have to spend just as much time slowing down as you did speeding up. Hence, the flip at the halfway point. It's also possible to have a coast period in the middle, though obviously you wouldn't speed up as much, and would consume less fuel.

\par
The time for a torch burn is:
\[t = \sqrt{\frac{d}{a}}\]

The fuel consumption for a torch burn is:
\[f = 2F\sqrt{\frac{d}{a}}\]

The maximum velocity during a torch burn is:
\[v = a\sqrt{\frac{d}{a}}\]

\textit{\newline where}
\par\textit{a is the acceleration rate during the burn,}
\par\textit{d is the distance to be covered during the burn,}
\par\textit{F is the fuel consumption rate for the engine at the given accelaration.}

\subsubsection{FTL Jumps}
\par
FTL jumps in \getTitle\, are called \textit{Punch-Tunnel} Jumps. A Punch-Tunnel jump is a Dodge \#2\footnote{\href{http://www.projectrho.com/public_html/rocket/fasterlight.php\#commonhandwaves}{See the section on Common Hand-Waves in the Faster-Than-Light article on Project Rho}}-type FTL drive. It reduces the amount of space needed to travel, meaning the ship has to travel a shorter distance thanks to the FTL technology.

\par
A punch-tunnel is an Einstein-Rosen bridge created by the ship from its location to its destination. A 3D hole appears in space in front of the ship, at first stationary to the ship's frame of reference, and the hole moves \textit{through} the ship while the ship remains on its current trajectory. As the hole passes through the ship, the entire ship's matter is moved, slice-by-slice, to the new location. This process happens in roughly a tenth of a second. The time in between the two locations is 20 miliseconds.

\par
Because the slices are 3D out of the 4D shape that all matter has, the ship appears to suddenly collapse along its long axis until it dissapears. Likewise, when it appears, it seems to extend from a thin slice into its full length.

\newcommand{\calculationFactor}{12}
\newcommand{\energyLightyearFactor}{8^d}

\paragraph{Time and Calculations}All jumps are instantanious from the ship's perspective, and forty-one minutes from a typical outsider's reference frame, no matter the distance. Calculations take longer time to complete for longer distances. Per lightyear travelled, it takes \calculationFactor\, minutes to complete the calculation. That is, given \textit{d} lightyears travelled, the time \textit{t} it takes to complete the calculation in minutes is given by:

\[t = 12\times d\]

\par This calculation results in a destination accuracy of about 1 AU\footnote{1 Astronomical Unit, the distance between the Earth and the sun.}, give or take. Note that a hole can only open in empty space, making jump collisions impossible. You cannot accidentally jump to within the center of another object. Instead, the hole opens nearby. This can still be dangerous. Smaller objects cannot block a hole from opening, but can fall through the punch-tunnel, causing a collision.

\paragraph{Accuracy increases}The accuracy of a destination's location can be increased. This calculation takes more time, of course.

\paragraph{Energy Usage}A jump takes up more energy the farther away it is. The energy a jump takes, where \textit{d} is the distance in lightyears and \textit{e} is the resulting energy cost in megajoules is given by:

\par
\[e = \energyLightyearFactor\, MW\]

\paragraph{Restrictions}Like many mechanics in \getTitle, certain restrictions and failures are story-driven. There are no story-mechanical reasons a jump cannot work, but the \gm may restrict the ability to punch-jump under certain circumstances. Here are some reasons the FTL drive may not work:

\begin{itemize}
	\item The ship has tidal forces applying to it, due to extreme gravity like a black hole.
	\item A strong magnetic field distrupts the ability of the FTL engine to maintain a cohesive punch-hole, and the ship must move away from the field.
	\item The ship has overheated, and must vent heat into space via thermal exhuast or radiators.
\end{itemize}

\par It is important to remember that the \gm\, is not trying to defeat the players, but co-operatively build the story with them. Restrictions should make thematic sense and further the story. Likewise, the restrictions should make sense in-universe, and not exists simply to keep the players in one place.

%http://www.projectrho.com/public_html/rocket/fasterlight.php#commonhandwaves

\subsection{Equation Cheat Sheet}

\subsection{Energy Source}
\par
Typically, an energy source for a ship will be nuclear. There are exceptions, though.

\par
In \getTitle , energy is measured in Megawatts, abbreviated as MW.

\subsubsection{Fission}

\par
Ah, fission. Baby's first nuclear reactor. Nuclear reactors on present-day earth are fission reactors. A result of the Manhattan project, fission was researched orginally to make bombs to help the United States blast entire cities away in one go. Fission reactors are heavy and not quite as fun as other atomic toys.

\par
Fission reactors use fuel rods of Uranium-235, Uranium-233, or Plutonium-239\footnote{This is the stuff used in atom bombs. Maybe reactors on \textsuperscript{239}P are illegal in many locales. Campaign idea?}. \getTitle\, doesn't much distinguish between these types of fuel for practical purposes, other than that these materials may have different availabilities in different locations, different prices, and that a given reactor will only take the fuel it was designed for.

\par
Fission reactors produce waste rods when they are spent. They still contain around 85\% fuel, but too sparse (i.e. not dense enough) to continue using as fuel. Locals do not take kindly to the dumping of spent fuel rods. The materials can have a half life in the hundreds of thousands of years, and produce dangerous amounts of radiation. Note also that though fuel may be spent slowly, a minimum "critical mass" of material must be packed into a sphere, or else the reaction will fizzle out. This sphere is less than fifteen centimeters in diameter.

\paragraph{Fission Fuel Consumption Chart}\footnote{Sourced largely from \href{http://www.projectrho.com/public_html/rocket/atomicfuel.php}{Project Rho: Atomic Fuel}}
\paragraph{}

\vspace{1\baselineskip}
\begin{tabular}{| r | r | r | r | r |}

\hline

Fuel & MeV & J/Kg & 1000 MW Burn & Critical Mass\\

\hline

\textsuperscript{235}U & 202.5 MeV & $83.14 \times 10^{12}$ J/Kg & 0.00001208 Kg/s & 52 Kg\\

\textsuperscript{233}U & 197.9 MeV & $81.95 \times 10^{12}$ J/Kg & 0.00001220 Kg/s & 15 Kg\\

\textsuperscript{239}P & 207.1 MeV & $83.61 \times 10^{12}$ J/Kg & 0.00001196 Kg/s & 10 Kg\\

Lazy Estimations & 200 MeV & $8 \times 10^{13}$ J/Kg & 0.00001 Kg/s & 10-50 Kg\\

\hline

\end{tabular}
\vspace{1\baselineskip}

\par
\textit{Where:}
\par
\textit{MeV is millions of electron-volts,}
\par
\textit{J/Kg is Joules produced per kilogram of the fuel, and}
\par
\textit{1000 MW Burn is grams consumed per second to produce 1000 MW.}

\paragraph{}
And the fuel burn rate total is:

$r = p \times o$

\par
\textit{Where:}
\par
\textit{r is the nuclear fuel burn rate in Kg/s,}
\par
\textit{p is the required power that you wish to generate, and}
\par
\textit{o is the energy output of the fuel, measured in joules/kilogram (the J/Kg column).}

\paragraph{Going Subcritical}As nuclear fuel is spent to generate power, keep track of how much fuel is spent. When the nuclear mass falls under the required critical mass, then the reactor will go \textit{subcritical}, meaning that it will no longer effectively generate power. The spent fuel rods must be removed, and rich fuel rods must be put in the reactor before it will continue generating power.

\subsubsection{Fusion}

\textit{Yet to research and write.}

\subsubsection{Exotic}

\textit{Yet to research and write.}

\subsubsection{Fuel Breeders}

Though not a direct power source, a breeder reactor can turn inert material into fissile material for use in other reactors. Spent fuel rods can usually be re-enriched for repeated use. More information on this in later editions.

\subsection{Engines}

There are multiple types of engines in \getTitle , some of them more conventional, and others near- or far-future technologies. The handwavium ones are more powerful.

\subsubsection{NERVA}

\textit{Yet to research and write.}

\subsubsection{Nuclear Pulse}

\textit{Yet to research and write.}

\subsubsection{Magneto Inertial Fusion}

\textit{Yet to research and write.}

\subsubsection{Nuclear Saltwater}

\textit{Yet to research and write.}

\subsubsection{Ion}

\textit{Yet to research and write.}

\subsubsection{Aerospike}

\textit{Yet to research and write.}

\subsubsection{Sabre}

\textit{Yet to research and write.}

\subsubsection{Anti-Matter}

\textit{Yet to research and write.}

\subsubsection{Exotic Matter}

\textit{Yet to research and write.}

\subsection{FTL Engine}

\textit{Yet to research and write.}

\subsection{Structure}

\textit{Yet to research and write.}

\subsubsection{Robust}

\subsubsection{Medium}

\subsubsection{Lightweight}

\subsubsection{Foil}

\subsubsection{Plate}

\subsubsection{Reinforced}

\subsection{Armaments}

\subsubsection{Lasers}
\par
These are basically useless for combat. They can ablate a little material, but not fast enough to save you when you're in a bind. But sure, you can have lasers.

\par
The one thing lasers \textit{can} do is blind EM-based sensors. You may be able to disrupt sensors on another ship, but it's also a beacon helping them see your exact location.

%[Laser info here]

\subsubsection{Railguns}
\par
This, now \textit{this} is what you want to use. Nothing can ruin someone's day faster than hundreds of metal slugs slamming into their ship at thousands of meters per second.

\par
Be cautious, since some planets may not take kindly to you generating debris in important orbital lanes.

%[Railgun info here]

\subsubsection{Missiles}

\textit{Yet to research and write.}
%[Missile info here]

\subsubsection{Magnetic Pulse}

\textit{Yet to research and write.}
%[MP info here]

\subsubsection{Anti-Missile Missiles}

\subsection{Sensors}

\subsubsection{Radio}
Pictures, spectro-analysis, radio telescopes.

\subsubsection{Astrogation}

\textit{Yet to research and write.}

\subsubsection{Telescopes}

\textit{Yet to research and write.}

\subsubsection{Docking Radar}

\textit{Yet to research and write.}

\subsubsection{Gravitometer}

\textit{Yet to research and write.}

\subsection{Ansible}
\par
An ansible is similar to a radio, but its signal propagates faster than the speed of light\footnote{In real life, this would either mean that relativity would be violated, since information would be travelling faster than light, or that causality would be violated, since it would be possible to recieve news before it even \textit{happened}, under certain conditions. But with FTL travel, we've already broken one of those two.}. This makes it possible to communicate with other ships and bases several lightyears away without having to wait several years for that light to travel. In \getTitle, there are two types of ansibles.

\subsubsection{Type A}
\par
The Type-A ansible can communicate with a constant round-trip time of seven minutes to any other Type-A ansible you know of, within twenty lightyears. The signal can be repeated from base to base, of course, so any non-isolated base of your faction within twenty lightyears of you can probably carry the message further, with additional delays.

\subsubsection{Type B}
\par
A Type-B ansible can communicate instantly (or with arbitrarily low latency) to a \textit{paired} Type-B from any distance. Unfortunately, the link between two Type-Bs only lasts seven days \textit{relative to each ansible's time}.

\par
For example, if ansible 1 is at 'normal' time, but ansible 2 is experiencing time at an accelerated rate, then the ansibles fall out of sync when ansible 2 experiences the passage of about seven days. The link is destroyed when one ansible experiences seven days first.

\par
It takes fifteen minutes to sync two Type-Bs together, and they must be physically next to each other, in the same frame of reference\footnote{This means they should be under the same gravity and both moving the same speed. This will not be a problem if both ansibles are on the same ship, or on two ships that are docked together.}.

\par
Typically, shuttles of a ship will have Type-Bs on them, each with a matching Type-B on the main ship. Even being on two different planets in the same system, radio signals can take hours.